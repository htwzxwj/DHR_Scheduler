\documentclass{article}
\usepackage{amsmath, amssymb}
\usepackage{fontspec}
\usepackage{xeCJK}
\setCJKmainfont{SimSun} % or another installed Chinese font

\begin{document}

设有 $n$ 个不同的标签 $y_1, y_2, \dots, y_n$,每个标签对应的权重和为 $w_i > 0$。

定义平均权重:
\[
\bar{w} = \frac{1}{n} \sum_{i=1}^{n} w_i
\]

定义方差:
\[
\mathrm{Var}(w) =
\frac{1}{n} \sum_{i=1}^{n} \bigl(w_i - \bar{w}\bigr)^2
\]

定义最大可能方差:
\[
\mathrm{MaxVar} =
\bar{w}^2 (n-1)
\]

则返回的归一化方差 $H$ 定义为:
\[
H =
\begin{cases}
0, & n \leq 1 \\[10pt]
\displaystyle
\frac{\mathrm{Var}(w)}{\mathrm{MaxVar}}, & n > 1 \text{ 且 } \mathrm{MaxVar} > 0 \\[10pt]
0, & \mathrm{MaxVar} = 0
\end{cases}
\]

即:
\[
H =
\begin{cases}
0, & n \leq 1 \\[10pt]
\displaystyle
\frac{
\frac{1}{n} \sum_{i=1}^{n} (w_i - \bar{w})^2
}{
\bar{w}^2 (n-1)
}, & n > 1 \text{ 且 } \bar{w}^2 (n-1) > 0 \\[10pt]
0, & \text{否则}
\end{cases}
\]

\section*{Beta Accuracy 计算流程}

我们希望估计一个执行单元输出正确的概率 \( p \),给定最近 \( N \) 次输出的观测结果。

\subsection*{输入}

观测序列:
\[
X = (x_1, x_2, \dots, x_N), \quad x_i \in \{0,1\}
\]

其中:
\[
x_i =
\begin{cases} 
1, & \text{第 \(i\) 次输出正确} \\[6pt]
0, & \text{第 \(i\) 次输出错误}
\end{cases}
\]

正确次数:
\[
c = \sum_{i=1}^{N} x_i
\]

错误次数:
\[
N-c
\]

\subsection*{先验假设}

假设 \( p \) 的先验分布为均匀分布:
\[
p \sim \text{Beta}(1,1)
\]

表示对 \( p \) 没有先验偏好。

\subsection*{后验更新}

根据 Beta-Bernoulli 共轭性,经过 \( c \) 次正确和 \( N-c \) 次错误后,\( p \) 的后验分布为:
\[
p \mid X \sim \text{Beta}(a,b)
\]

其中:
\[
a = c+1, \quad b = (N-c)+1
\]

\subsection*{点估计}

取 Beta 后验分布的均值作为 \( p \) 的估计值:
\[
\hat{p}_{\text{beta}} =
\mathbb{E}[p \mid X] =
\frac{a}{a+b} =
\frac{c+1}{N+2}
\]

\subsection*{流程总结}

\begin{enumerate}
    \item 统计最近窗口 \( N \) 次观测中的正确次数:
    \[
    c = \sum_{i=1}^{N} x_i
    \]
    \item 计算 Beta 分布的参数:
    \[
    a = c+1, \quad b = N-c+1
    \]
    \item 计算 Beta 分布的均值:
    \[
    \hat{p}_{\text{beta}} =
    \frac{a}{a+b} =
    \frac{c+1}{N+2}
    \]
\end{enumerate}

\section*{执行体权重动态更新机制}

设执行体集合为 \( U = \{ u_1, u_2, \dots, u_M \} \),每个执行体 \( u_i \) 具有以下状态变量:
\begin{itemize}
  \item 当前权重 \( w_i \in [0,1] \),反映其在融合决策中的可信度;
  \item 当前输出标签 \( r_i \);
  \item 当前活跃状态由指示函数 \(\delta_i \in \{0,1\}\) 表示。
\end{itemize}

设当前系统融合输出为 \( r^* \),且设定一个固定的权重调整系数 \( \lambda \in (0,1) \) (称为衰减/增强因子)。

\subsection*{正确性判定}

定义执行体 \( u_i \) 在当前时刻输出的正确性指标:
\[
\mathrm{correct}_i =
\begin{cases}
1, & \text{若 } r_i = r^* \text{ 且 } r^* \neq \texttt{SCHEDULED\_SIGNAL} \\[8pt]
0, & \text{否则}
\end{cases}
\]

其中 \texttt{SCHEDULED\_SIGNAL} 是特殊的调度指令标签,不作为标准的正确输出。

\subsection*{权重更新规则}

执行体的权重更新满足如下规则:
\[
w_i^{(t+1)} =
\begin{cases}
\min\bigl(1, \ w_i^{(t)} \cdot (1+\lambda) \bigr), & \text{若 } \mathrm{correct}_i = 1 \\[10pt]
\max\bigl(0, \ w_i^{(t)} \cdot (1-\lambda) \bigr), & \text{若 } \mathrm{correct}_i = 0
\end{cases}
\]

即,当执行体输出正确时,其权重按比例增加,并且不超过上界 1;当输出错误时,其权重按比例衰减,并且不低于下界 0。

\subsection*{合并形式}

上述更新规则可合并写为:
\[
w_i^{(t+1)} =
\mathrm{Proj}_{[0,1]} \Bigl( \, w_i^{(t)} \cdot \bigl( 1 + \lambda \cdot (2\, \mathrm{correct}_i -1) \bigr) \, \Bigr)
\]

其中投影算子 \(\mathrm{Proj}_{[0,1]}(x)\) 定义为:
\[
\mathrm{Proj}_{[0,1]}(x) =
\min\bigl(1, \max(0, x) \bigr)
\]

用于保证更新后的权重始终处于闭区间 \([0,1]\) 之内。

\subsection*{机制说明}

该更新机制通过指数形式的递增/递减,动态调整执行体的权重:
\begin{itemize}
  \item 当输出正确时,权重增强,反映其贡献的可靠性;
  \item 当输出错误时,权重衰减,降低其对决策的影响;
  \item 权重始终保持在区间 \([0,1]\) 内,便于解释与比较。
\end{itemize}

该机制兼顾了短期表现的敏感性和长期稳定性,有助于在系统中动态筛选高可信的执行体参与融合。


\end{document}
